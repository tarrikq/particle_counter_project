% Options for packages loaded elsewhere
\PassOptionsToPackage{unicode}{hyperref}
\PassOptionsToPackage{hyphens}{url}
%
\documentclass[
]{article}
\usepackage{amsmath,amssymb}
\usepackage{lmodern}
\usepackage{ifxetex,ifluatex}
\ifnum 0\ifxetex 1\fi\ifluatex 1\fi=0 % if pdftex
  \usepackage[T1]{fontenc}
  \usepackage[utf8]{inputenc}
  \usepackage{textcomp} % provide euro and other symbols
\else % if luatex or xetex
  \usepackage{unicode-math}
  \defaultfontfeatures{Scale=MatchLowercase}
  \defaultfontfeatures[\rmfamily]{Ligatures=TeX,Scale=1}
\fi
% Use upquote if available, for straight quotes in verbatim environments
\IfFileExists{upquote.sty}{\usepackage{upquote}}{}
\IfFileExists{microtype.sty}{% use microtype if available
  \usepackage[]{microtype}
  \UseMicrotypeSet[protrusion]{basicmath} % disable protrusion for tt fonts
}{}
\makeatletter
\@ifundefined{KOMAClassName}{% if non-KOMA class
  \IfFileExists{parskip.sty}{%
    \usepackage{parskip}
  }{% else
    \setlength{\parindent}{0pt}
    \setlength{\parskip}{6pt plus 2pt minus 1pt}}
}{% if KOMA class
  \KOMAoptions{parskip=half}}
\makeatother
\usepackage{xcolor}
\IfFileExists{xurl.sty}{\usepackage{xurl}}{} % add URL line breaks if available
\IfFileExists{bookmark.sty}{\usepackage{bookmark}}{\usepackage{hyperref}}
\hypersetup{
  pdftitle={Particle counter seperate loops WIP},
  pdfauthor={Tarrik Quneibi},
  hidelinks,
  pdfcreator={LaTeX via pandoc}}
\urlstyle{same} % disable monospaced font for URLs
\usepackage[margin=1in]{geometry}
\usepackage{color}
\usepackage{fancyvrb}
\newcommand{\VerbBar}{|}
\newcommand{\VERB}{\Verb[commandchars=\\\{\}]}
\DefineVerbatimEnvironment{Highlighting}{Verbatim}{commandchars=\\\{\}}
% Add ',fontsize=\small' for more characters per line
\usepackage{framed}
\definecolor{shadecolor}{RGB}{248,248,248}
\newenvironment{Shaded}{\begin{snugshade}}{\end{snugshade}}
\newcommand{\AlertTok}[1]{\textcolor[rgb]{0.94,0.16,0.16}{#1}}
\newcommand{\AnnotationTok}[1]{\textcolor[rgb]{0.56,0.35,0.01}{\textbf{\textit{#1}}}}
\newcommand{\AttributeTok}[1]{\textcolor[rgb]{0.77,0.63,0.00}{#1}}
\newcommand{\BaseNTok}[1]{\textcolor[rgb]{0.00,0.00,0.81}{#1}}
\newcommand{\BuiltInTok}[1]{#1}
\newcommand{\CharTok}[1]{\textcolor[rgb]{0.31,0.60,0.02}{#1}}
\newcommand{\CommentTok}[1]{\textcolor[rgb]{0.56,0.35,0.01}{\textit{#1}}}
\newcommand{\CommentVarTok}[1]{\textcolor[rgb]{0.56,0.35,0.01}{\textbf{\textit{#1}}}}
\newcommand{\ConstantTok}[1]{\textcolor[rgb]{0.00,0.00,0.00}{#1}}
\newcommand{\ControlFlowTok}[1]{\textcolor[rgb]{0.13,0.29,0.53}{\textbf{#1}}}
\newcommand{\DataTypeTok}[1]{\textcolor[rgb]{0.13,0.29,0.53}{#1}}
\newcommand{\DecValTok}[1]{\textcolor[rgb]{0.00,0.00,0.81}{#1}}
\newcommand{\DocumentationTok}[1]{\textcolor[rgb]{0.56,0.35,0.01}{\textbf{\textit{#1}}}}
\newcommand{\ErrorTok}[1]{\textcolor[rgb]{0.64,0.00,0.00}{\textbf{#1}}}
\newcommand{\ExtensionTok}[1]{#1}
\newcommand{\FloatTok}[1]{\textcolor[rgb]{0.00,0.00,0.81}{#1}}
\newcommand{\FunctionTok}[1]{\textcolor[rgb]{0.00,0.00,0.00}{#1}}
\newcommand{\ImportTok}[1]{#1}
\newcommand{\InformationTok}[1]{\textcolor[rgb]{0.56,0.35,0.01}{\textbf{\textit{#1}}}}
\newcommand{\KeywordTok}[1]{\textcolor[rgb]{0.13,0.29,0.53}{\textbf{#1}}}
\newcommand{\NormalTok}[1]{#1}
\newcommand{\OperatorTok}[1]{\textcolor[rgb]{0.81,0.36,0.00}{\textbf{#1}}}
\newcommand{\OtherTok}[1]{\textcolor[rgb]{0.56,0.35,0.01}{#1}}
\newcommand{\PreprocessorTok}[1]{\textcolor[rgb]{0.56,0.35,0.01}{\textit{#1}}}
\newcommand{\RegionMarkerTok}[1]{#1}
\newcommand{\SpecialCharTok}[1]{\textcolor[rgb]{0.00,0.00,0.00}{#1}}
\newcommand{\SpecialStringTok}[1]{\textcolor[rgb]{0.31,0.60,0.02}{#1}}
\newcommand{\StringTok}[1]{\textcolor[rgb]{0.31,0.60,0.02}{#1}}
\newcommand{\VariableTok}[1]{\textcolor[rgb]{0.00,0.00,0.00}{#1}}
\newcommand{\VerbatimStringTok}[1]{\textcolor[rgb]{0.31,0.60,0.02}{#1}}
\newcommand{\WarningTok}[1]{\textcolor[rgb]{0.56,0.35,0.01}{\textbf{\textit{#1}}}}
\usepackage{graphicx}
\makeatletter
\def\maxwidth{\ifdim\Gin@nat@width>\linewidth\linewidth\else\Gin@nat@width\fi}
\def\maxheight{\ifdim\Gin@nat@height>\textheight\textheight\else\Gin@nat@height\fi}
\makeatother
% Scale images if necessary, so that they will not overflow the page
% margins by default, and it is still possible to overwrite the defaults
% using explicit options in \includegraphics[width, height, ...]{}
\setkeys{Gin}{width=\maxwidth,height=\maxheight,keepaspectratio}
% Set default figure placement to htbp
\makeatletter
\def\fps@figure{htbp}
\makeatother
\setlength{\emergencystretch}{3em} % prevent overfull lines
\providecommand{\tightlist}{%
  \setlength{\itemsep}{0pt}\setlength{\parskip}{0pt}}
\setcounter{secnumdepth}{-\maxdimen} % remove section numbering
\ifluatex
  \usepackage{selnolig}  % disable illegal ligatures
\fi

\title{Particle counter seperate loops WIP}
\author{Tarrik Quneibi}
\date{6/3/2021}

\begin{document}
\maketitle

install and read in al necessary packages. Some of these may not be
used, I need to go back through and check which ones are absolutely
necessary.

\begin{Shaded}
\begin{Highlighting}[]
\NormalTok{knitr}\SpecialCharTok{::}\NormalTok{opts\_chunk}\SpecialCharTok{$}\FunctionTok{set}\NormalTok{(}\AttributeTok{echo =} \ConstantTok{TRUE}\NormalTok{)}

\DocumentationTok{\#\#install packages}
\DocumentationTok{\#\#install.packages("ggplot2")}
\DocumentationTok{\#\#install.packages("reshape2")}
\DocumentationTok{\#\#install.packages("dplyr")}
\DocumentationTok{\#\#install.packages("tidyr")}
\DocumentationTok{\#\#install.packages("chron")}
\DocumentationTok{\#\#install.packages("anytime")}
\DocumentationTok{\#\#install.packages("lubridate")}
\DocumentationTok{\#\#install.packages("tidyverse")}

\FunctionTok{library}\NormalTok{(ggplot2)}
\end{Highlighting}
\end{Shaded}

\begin{verbatim}
## Warning: package 'ggplot2' was built under R version 4.0.5
\end{verbatim}

\begin{Shaded}
\begin{Highlighting}[]
\FunctionTok{library}\NormalTok{(reshape2)}
\end{Highlighting}
\end{Shaded}

\begin{verbatim}
## Warning: package 'reshape2' was built under R version 4.0.5
\end{verbatim}

\begin{Shaded}
\begin{Highlighting}[]
\FunctionTok{library}\NormalTok{(dplyr)}
\end{Highlighting}
\end{Shaded}

\begin{verbatim}
## Warning: package 'dplyr' was built under R version 4.0.5
\end{verbatim}

\begin{verbatim}
## 
## Attaching package: 'dplyr'
\end{verbatim}

\begin{verbatim}
## The following objects are masked from 'package:stats':
## 
##     filter, lag
\end{verbatim}

\begin{verbatim}
## The following objects are masked from 'package:base':
## 
##     intersect, setdiff, setequal, union
\end{verbatim}

\begin{Shaded}
\begin{Highlighting}[]
\FunctionTok{library}\NormalTok{(tidyr)}
\end{Highlighting}
\end{Shaded}

\begin{verbatim}
## Warning: package 'tidyr' was built under R version 4.0.5
\end{verbatim}

\begin{verbatim}
## 
## Attaching package: 'tidyr'
\end{verbatim}

\begin{verbatim}
## The following object is masked from 'package:reshape2':
## 
##     smiths
\end{verbatim}

\begin{Shaded}
\begin{Highlighting}[]
\FunctionTok{library}\NormalTok{(chron)}
\end{Highlighting}
\end{Shaded}

\begin{verbatim}
## Warning: package 'chron' was built under R version 4.0.5
\end{verbatim}

\begin{Shaded}
\begin{Highlighting}[]
\FunctionTok{library}\NormalTok{(anytime)}
\end{Highlighting}
\end{Shaded}

\begin{verbatim}
## Warning: package 'anytime' was built under R version 4.0.5
\end{verbatim}

\begin{Shaded}
\begin{Highlighting}[]
\FunctionTok{library}\NormalTok{(lubridate)}
\end{Highlighting}
\end{Shaded}

\begin{verbatim}
## Warning: package 'lubridate' was built under R version 4.0.5
\end{verbatim}

\begin{verbatim}
## 
## Attaching package: 'lubridate'
\end{verbatim}

\begin{verbatim}
## The following objects are masked from 'package:chron':
## 
##     days, hours, minutes, seconds, years
\end{verbatim}

\begin{verbatim}
## The following objects are masked from 'package:base':
## 
##     date, intersect, setdiff, union
\end{verbatim}

\begin{Shaded}
\begin{Highlighting}[]
\FunctionTok{library}\NormalTok{(tidyverse)}
\end{Highlighting}
\end{Shaded}

\begin{verbatim}
## Warning: package 'tidyverse' was built under R version 4.0.5
\end{verbatim}

\begin{verbatim}
## -- Attaching packages --------------------------------------- tidyverse 1.3.1 --
\end{verbatim}

\begin{verbatim}
## v tibble  3.1.1     v stringr 1.4.0
## v readr   1.4.0     v forcats 0.5.1
## v purrr   0.3.4
\end{verbatim}

\begin{verbatim}
## Warning: package 'tibble' was built under R version 4.0.5
\end{verbatim}

\begin{verbatim}
## Warning: package 'readr' was built under R version 4.0.5
\end{verbatim}

\begin{verbatim}
## Warning: package 'purrr' was built under R version 4.0.5
\end{verbatim}

\begin{verbatim}
## Warning: package 'stringr' was built under R version 4.0.5
\end{verbatim}

\begin{verbatim}
## Warning: package 'forcats' was built under R version 4.0.5
\end{verbatim}

\begin{verbatim}
## -- Conflicts ------------------------------------------ tidyverse_conflicts() --
## x lubridate::as.difftime() masks base::as.difftime()
## x lubridate::date()        masks base::date()
## x lubridate::days()        masks chron::days()
## x dplyr::filter()          masks stats::filter()
## x lubridate::hours()       masks chron::hours()
## x lubridate::intersect()   masks base::intersect()
## x dplyr::lag()             masks stats::lag()
## x lubridate::minutes()     masks chron::minutes()
## x lubridate::seconds()     masks chron::seconds()
## x lubridate::setdiff()     masks base::setdiff()
## x lubridate::union()       masks base::union()
## x lubridate::years()       masks chron::years()
\end{verbatim}

setting the working directory Calling in all files that end in .csv.
Creating empty lists for the particle data(table\_p), flow data
(table\_d), and file names for both (p\_titles, f\_titles). creating
vectors for the solumn names which will be applied after each dataframe
is cleaned.

\begin{verbatim}
##setting the working directory
setwd("U:/public/ADMIN/WQIntern TQuneibi/R code/particle counter data/")

##adds all .csv files in the working directory into a list
file_list <- list.files(pattern = ".csv")

##creating empty lists to later add variables into
table_p <-list()
table_f <- list()
p_titles <- list()
f_titles <- list()
file_titles <- list()

##list of column names for particle data and flow data
p_columns <- c("Date","Bin 1", "Bin 2", "Bin 3", "Bin 4", "Bin 5", "Bin 6", "Bin 7", "Bin 8", "Bin 9")
f_columns <- c("Date", "Flow", "Turbidity")
\end{verbatim}

This loop will take in only files from the file\_list that do not have
the word ``Flow'' in the file name. This is how the particle counter
data will be found. The loop begins by changing the name of the file so
it is easier to read and then stores the name in the vector (p\_titles)
The file is then read and the columns of interest (columns 2-12) are
kept while the rest is removed. The date and time columns are then
combined into one column and changed to date-time class. The old time
column is deleted. The date column is then rounded to the nearest 15
minute interval to match the flow data. The data and file name are
combined and placed into table\_p to create a list of all the files.
Outside of the for loop, the list of files are iterated through to
remove any null entries and the file name is then applied to each
dataframe.

\begin{verbatim}
##iteration variable
i <- 0
##loop through directory to pull out all particle data
for (file in file_list){
  i=i+1
 if(!grepl("Flow", file)){
   
##removes the .csv and date from the end of each file name and stores the name in a list
    titles <- substring(file, 1, nchar(file) - 13)
    p_titles[i] <- titles
    
##reading in the file and taking only the important columns
    p_data <- read.csv(file)
    p_data <- p_data[, 2:12]
    
##combines the date and time columns back into a single column with the correct format. 
##Also changes the class from CHAR to Date-Time
    p_data$Date <- as.POSIXct(paste(p_data$Date, p_data$Time), format="%m/%d/%Y %H:%M")
    
##deleting the old time column
    p_data[ , c('Time')] <- list(NULL)
    
##the time was rounded to the nearest 15 minute mark so that it could be accurately joined with the flow data
    p_data$Date <- round_date(p_data$Date, unit = "15 minutes")
    
##combines the file name with the data associated with it to create a variable
    assign(titles, p_data)
    table_p <- append(table_p, list(p_data))
    names(table_p[i]) <- titles
 }}

##removes all the null entries in both lists and then names each data frame by its file name
table_p <- table_p[-which(lapply(table_p,is.null) == T)]
p_titles <- p_titles[-which(lapply(p_titles,is.null) == T)]
names(table_p) <- p_titles
\end{verbatim}

This loop will take in only files from the file\_list that have the word
``Flow'' in the file name. This is how the flow data will be found. The
loop begins by changing the name of the file so it is easier to read and
then stores the name in the vector (f\_titles) The file is then read and
the columns of interest (columns 1-3) are kept while the rest is
removed. Every 3rd row is then taken and the rest removed to have a data
frame for each 15 minute interval (rather than every 5 minutes) The date
and time columns are then split into two columns and formatted to match
the particle counter data (MDY) The date and time columns are then
combined into one column and changed to date-time data. The date column
is then rounded to the nearest 15 minute interval to match the flow
data. The flow data column is then rounded to either 1 or 0 depending if
it is greater than or less than 0.01 The data and file name are combined
and placed into table\_f to create a list of all the files. Outside of
the for loop, the list of files are iterated through to remove any null
entries and the file name is then applied to each dataframe.

\begin{verbatim}
##iteration variable
i <- 0

##loop to pull all flow data from the directory
for (file in file_list){
  i=i+1
  if(grepl("Flow", file)){
    
##removes the .csv from the end of each file name
   titles <- substring(file, 1, nchar(file) - 13)
   f_titles[i] <- titles
   
##reading each file
   f_data <- read.csv(file)
   f_data <- f_data[ ,1:3]
   
##takes every third row so that only every 15 minutes is considered
   f_data = f_data[seq(1, nrow(f_data), 3), ]
   
##changes the first column name to date
   colnames(f_data)[1] <- "Date"
   
##seperates the date and time into two columns
   f_data <- f_data %>% separate(Date, c("Date", "Time"), " ")
   
#changes the date to a new format which matches the particle counter date format
   f_data$Date <- as.Date(f_data$Date, format = "%m/%d/%y")
   
##combines the date and time columns back into a single column with the correct format. 
##Also changes the class from CHAR to Date-Time
   f_data$Date <- as.POSIXct(paste(f_data$Date, f_data$Time), format="%Y-%m-%d %H:%M")
   
##deleting the old time column
   f_data[ , c('Time')] <- list(NULL)
   
##the time was rounded to the nearest 15 minute mark so that it could be accurately joined with the particle data
   f_data$Date <- round_date(f_data$Date, unit = "15 minutes")
   
##takes any flow data value above 0.01 and changes it to 1, while anything below 0.01 is changed to 0
   f_data[ , 2] <- ifelse(f_data[ ,2] > 0.01, 1, 0)
   
##combines the file name with the data associated with it to create a variable
    assign(titles, f_data)
    table_f <- append(table_f, list(f_data))
    names(table_f[i]) <- titles
  }
}

##removes all the null entries in both lists and then names each data frame by its file name
table_f <- table_f[-which(lapply(table_f,is.null) == T)]
f_titles <- f_titles[-which(lapply(f_titles,is.null) == T)]
names(table_f) <- f_titles
\end{verbatim}

\begin{verbatim}
                                                                                   
###final_data.join <- left_join(table_p[1], table_f[2], by = c("Date"))
\end{verbatim}

everything below here is to later be used inside the loop above

\#\#Joining the particle data and flow data by the data and time
final\_data.join \textless- left\_join(p\_data, f\_data, by =
c(``Date''))

\#\#////////////////////////////////////Dont run below code until the
data frames are joined correctly
\#\#////////////////////////////////////Will need to add a loop to go
through all runs for each filter to plot

\#Multiply flow data and particle data to get relevant data counts
\textless- final\_data.join{[} ,5:13{]} flow \textless-
final\_data.join{[} , 30{]} relevent\_data \textless- counts*flow

\#\#add the date back in to the data frame relevent\_data.time
\textless- cbind(p\_data\$Date, relevent\_data)
colnames(relevent\_data.time) \textless-
c(``Date'',``Bin1'',``Bin2'',``Bin3'',``Bin4'',``Bin5'',``Bin6'',``Bin7'',``Bin8'',``Bin9'')
final\_data \textless- melt(relevent\_data.time,id.vars=``Date'' )

\#\#plotting the data as a scatterplot ggplot(final\_data, aes(Date,
value, col=variable)) + geom\_point() + stat\_smooth() +
ggtitle(``Particle counts from 2/19/21-2/20/21'') + \# for the main
title xlab(``Date'') + \# for the x axis label ylab(``Counts'') + \# for
the y axis label scale\_x\_discrete(breaks = c(2/19/21, 2/20/21,\\
2/20/21 ))

\end{document}
