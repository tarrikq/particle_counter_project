% Options for packages loaded elsewhere
\PassOptionsToPackage{unicode}{hyperref}
\PassOptionsToPackage{hyphens}{url}
%
\documentclass[
]{article}
\usepackage{amsmath,amssymb}
\usepackage{lmodern}
\usepackage{ifxetex,ifluatex}
\ifnum 0\ifxetex 1\fi\ifluatex 1\fi=0 % if pdftex
  \usepackage[T1]{fontenc}
  \usepackage[utf8]{inputenc}
  \usepackage{textcomp} % provide euro and other symbols
\else % if luatex or xetex
  \usepackage{unicode-math}
  \defaultfontfeatures{Scale=MatchLowercase}
  \defaultfontfeatures[\rmfamily]{Ligatures=TeX,Scale=1}
\fi
% Use upquote if available, for straight quotes in verbatim environments
\IfFileExists{upquote.sty}{\usepackage{upquote}}{}
\IfFileExists{microtype.sty}{% use microtype if available
  \usepackage[]{microtype}
  \UseMicrotypeSet[protrusion]{basicmath} % disable protrusion for tt fonts
}{}
\makeatletter
\@ifundefined{KOMAClassName}{% if non-KOMA class
  \IfFileExists{parskip.sty}{%
    \usepackage{parskip}
  }{% else
    \setlength{\parindent}{0pt}
    \setlength{\parskip}{6pt plus 2pt minus 1pt}}
}{% if KOMA class
  \KOMAoptions{parskip=half}}
\makeatother
\usepackage{xcolor}
\IfFileExists{xurl.sty}{\usepackage{xurl}}{} % add URL line breaks if available
\IfFileExists{bookmark.sty}{\usepackage{bookmark}}{\usepackage{hyperref}}
\hypersetup{
  pdftitle={Particle counter seperate loops WIP},
  pdfauthor={Tarrik Quneibi},
  hidelinks,
  pdfcreator={LaTeX via pandoc}}
\urlstyle{same} % disable monospaced font for URLs
\usepackage[margin=1in]{geometry}
\usepackage{color}
\usepackage{fancyvrb}
\newcommand{\VerbBar}{|}
\newcommand{\VERB}{\Verb[commandchars=\\\{\}]}
\DefineVerbatimEnvironment{Highlighting}{Verbatim}{commandchars=\\\{\}}
% Add ',fontsize=\small' for more characters per line
\usepackage{framed}
\definecolor{shadecolor}{RGB}{248,248,248}
\newenvironment{Shaded}{\begin{snugshade}}{\end{snugshade}}
\newcommand{\AlertTok}[1]{\textcolor[rgb]{0.94,0.16,0.16}{#1}}
\newcommand{\AnnotationTok}[1]{\textcolor[rgb]{0.56,0.35,0.01}{\textbf{\textit{#1}}}}
\newcommand{\AttributeTok}[1]{\textcolor[rgb]{0.77,0.63,0.00}{#1}}
\newcommand{\BaseNTok}[1]{\textcolor[rgb]{0.00,0.00,0.81}{#1}}
\newcommand{\BuiltInTok}[1]{#1}
\newcommand{\CharTok}[1]{\textcolor[rgb]{0.31,0.60,0.02}{#1}}
\newcommand{\CommentTok}[1]{\textcolor[rgb]{0.56,0.35,0.01}{\textit{#1}}}
\newcommand{\CommentVarTok}[1]{\textcolor[rgb]{0.56,0.35,0.01}{\textbf{\textit{#1}}}}
\newcommand{\ConstantTok}[1]{\textcolor[rgb]{0.00,0.00,0.00}{#1}}
\newcommand{\ControlFlowTok}[1]{\textcolor[rgb]{0.13,0.29,0.53}{\textbf{#1}}}
\newcommand{\DataTypeTok}[1]{\textcolor[rgb]{0.13,0.29,0.53}{#1}}
\newcommand{\DecValTok}[1]{\textcolor[rgb]{0.00,0.00,0.81}{#1}}
\newcommand{\DocumentationTok}[1]{\textcolor[rgb]{0.56,0.35,0.01}{\textbf{\textit{#1}}}}
\newcommand{\ErrorTok}[1]{\textcolor[rgb]{0.64,0.00,0.00}{\textbf{#1}}}
\newcommand{\ExtensionTok}[1]{#1}
\newcommand{\FloatTok}[1]{\textcolor[rgb]{0.00,0.00,0.81}{#1}}
\newcommand{\FunctionTok}[1]{\textcolor[rgb]{0.00,0.00,0.00}{#1}}
\newcommand{\ImportTok}[1]{#1}
\newcommand{\InformationTok}[1]{\textcolor[rgb]{0.56,0.35,0.01}{\textbf{\textit{#1}}}}
\newcommand{\KeywordTok}[1]{\textcolor[rgb]{0.13,0.29,0.53}{\textbf{#1}}}
\newcommand{\NormalTok}[1]{#1}
\newcommand{\OperatorTok}[1]{\textcolor[rgb]{0.81,0.36,0.00}{\textbf{#1}}}
\newcommand{\OtherTok}[1]{\textcolor[rgb]{0.56,0.35,0.01}{#1}}
\newcommand{\PreprocessorTok}[1]{\textcolor[rgb]{0.56,0.35,0.01}{\textit{#1}}}
\newcommand{\RegionMarkerTok}[1]{#1}
\newcommand{\SpecialCharTok}[1]{\textcolor[rgb]{0.00,0.00,0.00}{#1}}
\newcommand{\SpecialStringTok}[1]{\textcolor[rgb]{0.31,0.60,0.02}{#1}}
\newcommand{\StringTok}[1]{\textcolor[rgb]{0.31,0.60,0.02}{#1}}
\newcommand{\VariableTok}[1]{\textcolor[rgb]{0.00,0.00,0.00}{#1}}
\newcommand{\VerbatimStringTok}[1]{\textcolor[rgb]{0.31,0.60,0.02}{#1}}
\newcommand{\WarningTok}[1]{\textcolor[rgb]{0.56,0.35,0.01}{\textbf{\textit{#1}}}}
\usepackage{graphicx}
\makeatletter
\def\maxwidth{\ifdim\Gin@nat@width>\linewidth\linewidth\else\Gin@nat@width\fi}
\def\maxheight{\ifdim\Gin@nat@height>\textheight\textheight\else\Gin@nat@height\fi}
\makeatother
% Scale images if necessary, so that they will not overflow the page
% margins by default, and it is still possible to overwrite the defaults
% using explicit options in \includegraphics[width, height, ...]{}
\setkeys{Gin}{width=\maxwidth,height=\maxheight,keepaspectratio}
% Set default figure placement to htbp
\makeatletter
\def\fps@figure{htbp}
\makeatother
\setlength{\emergencystretch}{3em} % prevent overfull lines
\providecommand{\tightlist}{%
  \setlength{\itemsep}{0pt}\setlength{\parskip}{0pt}}
\setcounter{secnumdepth}{-\maxdimen} % remove section numbering
\ifluatex
  \usepackage{selnolig}  % disable illegal ligatures
\fi

\title{Particle counter seperate loops WIP}
\author{Tarrik Quneibi}
\date{6/3/2021}

\begin{document}
\maketitle

install and read in all necessary packages. Some of these may not be
used, I need to go back through and check which ones are absolutely
necessary.

\begin{Shaded}
\begin{Highlighting}[]
\NormalTok{knitr}\SpecialCharTok{::}\NormalTok{opts\_chunk}\SpecialCharTok{$}\FunctionTok{set}\NormalTok{(}\AttributeTok{echo =} \ConstantTok{TRUE}\NormalTok{)}

\DocumentationTok{\#\#install packages}
\DocumentationTok{\#\#install.packages("ggplot2")}
\DocumentationTok{\#\#install.packages("reshape2")}
\DocumentationTok{\#\#install.packages("dplyr")}
\DocumentationTok{\#\#install.packages("tidyr")}
\DocumentationTok{\#\#install.packages("chron")}
\DocumentationTok{\#\#install.packages("anytime")}
\DocumentationTok{\#\#install.packages("lubridate")}
\DocumentationTok{\#\#install.packages("tidyverse")}

\FunctionTok{library}\NormalTok{(ggplot2)}
\FunctionTok{library}\NormalTok{(reshape2)}
\FunctionTok{library}\NormalTok{(dplyr)}
\FunctionTok{library}\NormalTok{(tidyr)}
\FunctionTok{library}\NormalTok{(chron)}
\FunctionTok{library}\NormalTok{(anytime)}
\FunctionTok{library}\NormalTok{(lubridate)}
\FunctionTok{library}\NormalTok{(tidyverse)}
\end{Highlighting}
\end{Shaded}

setting the working directory Calling in all files that end in .csv.
Creating empty lists for the particle data(table\_p), flow data
(table\_d), and file names for both (p\_titles, f\_titles). creating
vectors for the solumn names which will be applied after each dataframe
is cleaned.

\begin{Shaded}
\begin{Highlighting}[]
\DocumentationTok{\#\#setting the working directory}
\DocumentationTok{\#\#setwd("\textasciigrave{}/ R code/ particle counter data/")}

\DocumentationTok{\#\#adds all .csv files in the working directory into a list}
\NormalTok{file\_list }\OtherTok{\textless{}{-}} \FunctionTok{list.files}\NormalTok{(}\AttributeTok{pattern =} \StringTok{".csv"}\NormalTok{)}

\DocumentationTok{\#\#creating empty lists to later add variables into}
\NormalTok{table\_p }\OtherTok{\textless{}{-}}\FunctionTok{list}\NormalTok{()}
\NormalTok{table\_f }\OtherTok{\textless{}{-}} \FunctionTok{list}\NormalTok{()}
\NormalTok{p\_titles }\OtherTok{\textless{}{-}} \FunctionTok{list}\NormalTok{()}
\NormalTok{f\_titles }\OtherTok{\textless{}{-}} \FunctionTok{list}\NormalTok{()}
\NormalTok{file\_titles }\OtherTok{\textless{}{-}} \FunctionTok{list}\NormalTok{()}

\DocumentationTok{\#\#list of column names for particle data and flow data}
\NormalTok{p\_columns }\OtherTok{\textless{}{-}} \FunctionTok{c}\NormalTok{(}\StringTok{"Date"}\NormalTok{,}\StringTok{"Bin 1"}\NormalTok{, }\StringTok{"Bin 2"}\NormalTok{, }\StringTok{"Bin 3"}\NormalTok{, }\StringTok{"Bin 4"}\NormalTok{, }\StringTok{"Bin 5"}\NormalTok{, }\StringTok{"Bin 6"}\NormalTok{, }\StringTok{"Bin 7"}\NormalTok{, }\StringTok{"Bin 8"}\NormalTok{, }\StringTok{"Bin 9"}\NormalTok{)}
\NormalTok{f\_columns }\OtherTok{\textless{}{-}} \FunctionTok{c}\NormalTok{(}\StringTok{"Date"}\NormalTok{, }\StringTok{"Flow"}\NormalTok{, }\StringTok{"Turbidity"}\NormalTok{)}
\end{Highlighting}
\end{Shaded}

This loop will take in only files from the file\_list that do not have
the word ``Flow'' in the file name. This is how the particle counter
data will be found. The loop begins by changing the name of the file so
it is easier to read and then stores the name in the vector (p\_titles)
The file is then read and the columns of interest (columns 2-12) are
kept while the rest is removed. The date and time columns are then
combined into one column and changed to date-time class. The old time
column is deleted. The date column is then rounded to the nearest 15
minute interval to match the flow data. The data and file name are
combined and placed into table\_p to create a list of all the files.
Outside of the for loop, the list of files are iterated through to
remove any null entries and the file name is then applied to each
dataframe.

\begin{Shaded}
\begin{Highlighting}[]
\DocumentationTok{\#\#iteration variable}
\NormalTok{i }\OtherTok{\textless{}{-}} \DecValTok{0}
\DocumentationTok{\#\#loop through directory to pull out all particle data}
\ControlFlowTok{for}\NormalTok{ (file }\ControlFlowTok{in}\NormalTok{ file\_list)\{}
\NormalTok{  i}\OtherTok{=}\NormalTok{i}\SpecialCharTok{+}\DecValTok{1}
 \ControlFlowTok{if}\NormalTok{(}\SpecialCharTok{!}\FunctionTok{grepl}\NormalTok{(}\StringTok{"Flow"}\NormalTok{, file))\{}
   
\DocumentationTok{\#\#removes the .csv and date from the end of each file name and stores the name in a list}
\NormalTok{    titles }\OtherTok{\textless{}{-}} \FunctionTok{substring}\NormalTok{(file, }\DecValTok{1}\NormalTok{, }\FunctionTok{nchar}\NormalTok{(file) }\SpecialCharTok{{-}} \DecValTok{13}\NormalTok{)}
\NormalTok{    p\_titles[i] }\OtherTok{\textless{}{-}}\NormalTok{ titles}
    
\DocumentationTok{\#\#reading in the file and taking only the important columns}
\NormalTok{    p\_data }\OtherTok{\textless{}{-}} \FunctionTok{read.csv}\NormalTok{(file)}
\NormalTok{    p\_data }\OtherTok{\textless{}{-}}\NormalTok{ p\_data[, }\DecValTok{2}\SpecialCharTok{:}\DecValTok{12}\NormalTok{]}
    
\DocumentationTok{\#\#combines the date and time columns back into a single column with the correct format. }
\DocumentationTok{\#\#Also changes the class from CHAR to Date{-}Time}
\NormalTok{    p\_data}\SpecialCharTok{$}\NormalTok{Date }\OtherTok{\textless{}{-}} \FunctionTok{as.POSIXct}\NormalTok{(}\FunctionTok{paste}\NormalTok{(p\_data}\SpecialCharTok{$}\NormalTok{Date, p\_data}\SpecialCharTok{$}\NormalTok{Time), }\AttributeTok{format=}\StringTok{"\%m/\%d/\%Y \%H:\%M"}\NormalTok{)}
    
\DocumentationTok{\#\#deleting the old time column}
\NormalTok{    p\_data[ , }\FunctionTok{c}\NormalTok{(}\StringTok{\textquotesingle{}Time\textquotesingle{}}\NormalTok{)] }\OtherTok{\textless{}{-}} \FunctionTok{list}\NormalTok{(}\ConstantTok{NULL}\NormalTok{)}
    
\DocumentationTok{\#\#the time was rounded to the nearest 15 minute mark so that it could be accurately joined with the flow data}
\NormalTok{    p\_data}\SpecialCharTok{$}\NormalTok{Date }\OtherTok{\textless{}{-}} \FunctionTok{round\_date}\NormalTok{(p\_data}\SpecialCharTok{$}\NormalTok{Date, }\AttributeTok{unit =} \StringTok{"15 minutes"}\NormalTok{)}
    
\DocumentationTok{\#\#combines the file name with the data associated with it to create a variable}
    \FunctionTok{colnames}\NormalTok{(p\_data) }\OtherTok{\textless{}{-}}\NormalTok{ p\_columns}
    \FunctionTok{assign}\NormalTok{(titles, p\_data)}
\NormalTok{    table\_p }\OtherTok{\textless{}{-}} \FunctionTok{append}\NormalTok{(table\_p, }\FunctionTok{list}\NormalTok{(p\_data))}
    \FunctionTok{names}\NormalTok{(table\_p[i]) }\OtherTok{\textless{}{-}}\NormalTok{ titles}
    
\NormalTok{ \}\}}
\end{Highlighting}
\end{Shaded}

\begin{verbatim}
## Error in round_date(p_data$Date, unit = "15 minutes"): could not find function "round_date"
\end{verbatim}

\begin{Shaded}
\begin{Highlighting}[]
\DocumentationTok{\#\#removes all the null entries in both lists and then names each data frame by its file name}
\NormalTok{table\_p }\OtherTok{\textless{}{-}}\NormalTok{ table\_p[}\SpecialCharTok{{-}}\FunctionTok{which}\NormalTok{(}\FunctionTok{lapply}\NormalTok{(table\_p,is.null) }\SpecialCharTok{==}\NormalTok{ T)]}
\NormalTok{p\_titles }\OtherTok{\textless{}{-}}\NormalTok{ p\_titles[}\SpecialCharTok{{-}}\FunctionTok{which}\NormalTok{(}\FunctionTok{lapply}\NormalTok{(p\_titles,is.null) }\SpecialCharTok{==}\NormalTok{ T)]}
\FunctionTok{names}\NormalTok{(table\_p) }\OtherTok{\textless{}{-}}\NormalTok{ p\_titles}
\end{Highlighting}
\end{Shaded}

This loop will take in only files from the file\_list that have the word
``Flow'' in the file name. This is how the flow data will be found. The
loop begins by changing the name of the file so it is easier to read and
then stores the name in the vector (f\_titles) The file is then read and
the columns of interest (columns 1-3) are kept while the rest is
removed. Every 3rd row is then taken and the rest removed to have a data
frame for each 15 minute interval (rather than every 5 minutes) The date
and time columns are then split into two columns and formatted to match
the particle counter data (MDY) The date and time columns are then
combined into one column and changed to date-time data. The date column
is then rounded to the nearest 15 minute interval to match the flow
data. The flow data column is then rounded to either 1 or 0 depending if
it is greater than or less than 0.01 The data and file name are combined
and placed into table\_f to create a list of all the files. Outside of
the for loop, the list of files are iterated through to remove any null
entries and the file name is then applied to each dataframe.

\begin{Shaded}
\begin{Highlighting}[]
\DocumentationTok{\#\#iteration variable}
\NormalTok{i }\OtherTok{\textless{}{-}} \DecValTok{0}

\DocumentationTok{\#\#loop to pull all flow data from the directory}
\ControlFlowTok{for}\NormalTok{ (file }\ControlFlowTok{in}\NormalTok{ file\_list)\{}
\NormalTok{  i}\OtherTok{=}\NormalTok{i}\SpecialCharTok{+}\DecValTok{1}
  \ControlFlowTok{if}\NormalTok{(}\FunctionTok{grepl}\NormalTok{(}\StringTok{"Flow"}\NormalTok{, file))\{}
    
\DocumentationTok{\#\#removes the .csv from the end of each file name}
\NormalTok{   titles }\OtherTok{\textless{}{-}} \FunctionTok{substring}\NormalTok{(file, }\DecValTok{1}\NormalTok{, }\FunctionTok{nchar}\NormalTok{(file) }\SpecialCharTok{{-}} \DecValTok{13}\NormalTok{)}
\NormalTok{   f\_titles[i] }\OtherTok{\textless{}{-}}\NormalTok{ titles}
   
\DocumentationTok{\#\#reading each file}
\NormalTok{   f\_data }\OtherTok{\textless{}{-}} \FunctionTok{read.csv}\NormalTok{(file)}
\NormalTok{   f\_data }\OtherTok{\textless{}{-}}\NormalTok{ f\_data[ ,}\DecValTok{1}\SpecialCharTok{:}\DecValTok{3}\NormalTok{]}
   
\DocumentationTok{\#\#takes every third row so that only every 15 minutes is considered}
\NormalTok{   f\_data }\OtherTok{=}\NormalTok{ f\_data[}\FunctionTok{seq}\NormalTok{(}\DecValTok{1}\NormalTok{, }\FunctionTok{nrow}\NormalTok{(f\_data), }\DecValTok{3}\NormalTok{), ]}
   
\DocumentationTok{\#\#changes the first column name to date}
   \FunctionTok{colnames}\NormalTok{(f\_data)[}\DecValTok{1}\NormalTok{] }\OtherTok{\textless{}{-}} \StringTok{"Date"}
   
\DocumentationTok{\#\#seperates the date and time into two columns}
\NormalTok{   f\_data }\OtherTok{\textless{}{-}}\NormalTok{ f\_data }\SpecialCharTok{\%\textgreater{}\%} \FunctionTok{separate}\NormalTok{(Date, }\FunctionTok{c}\NormalTok{(}\StringTok{"Date"}\NormalTok{, }\StringTok{"Time"}\NormalTok{), }\StringTok{" "}\NormalTok{)}
   
\CommentTok{\#changes the date to a new format which matches the particle counter date format}
\NormalTok{   f\_data}\SpecialCharTok{$}\NormalTok{Date }\OtherTok{\textless{}{-}} \FunctionTok{as.Date}\NormalTok{(f\_data}\SpecialCharTok{$}\NormalTok{Date, }\AttributeTok{format =} \StringTok{"\%m/\%d/\%y"}\NormalTok{)}
   
\DocumentationTok{\#\#combines the date and time columns back into a single column with the correct format. }
\DocumentationTok{\#\#Also changes the class from CHAR to Date{-}Time}
\NormalTok{   f\_data}\SpecialCharTok{$}\NormalTok{Date }\OtherTok{\textless{}{-}} \FunctionTok{as.POSIXct}\NormalTok{(}\FunctionTok{paste}\NormalTok{(f\_data}\SpecialCharTok{$}\NormalTok{Date, f\_data}\SpecialCharTok{$}\NormalTok{Time), }\AttributeTok{format=}\StringTok{"\%Y{-}\%m{-}\%d \%H:\%M"}\NormalTok{)}
   
\DocumentationTok{\#\#deleting the old time column}
\NormalTok{   f\_data[ , }\FunctionTok{c}\NormalTok{(}\StringTok{\textquotesingle{}Time\textquotesingle{}}\NormalTok{)] }\OtherTok{\textless{}{-}} \FunctionTok{list}\NormalTok{(}\ConstantTok{NULL}\NormalTok{)}
   
\DocumentationTok{\#\#the time was rounded to the nearest 15 minute mark so that it could be accurately joined with the particle data}
\NormalTok{   f\_data}\SpecialCharTok{$}\NormalTok{Date }\OtherTok{\textless{}{-}} \FunctionTok{round\_date}\NormalTok{(f\_data}\SpecialCharTok{$}\NormalTok{Date, }\AttributeTok{unit =} \StringTok{"15 minutes"}\NormalTok{)}
   
\DocumentationTok{\#\#takes any flow data value above 0.01 and changes it to 1, while anything below 0.01 is changed to 0}
\NormalTok{   f\_data[ , }\DecValTok{2}\NormalTok{] }\OtherTok{\textless{}{-}} \FunctionTok{ifelse}\NormalTok{(f\_data[ ,}\DecValTok{2}\NormalTok{] }\SpecialCharTok{\textgreater{}} \FloatTok{0.01}\NormalTok{, }\DecValTok{1}\NormalTok{, }\DecValTok{0}\NormalTok{)}
   
\DocumentationTok{\#\#combines the file name with the data associated with it to create a variable}
   \FunctionTok{colnames}\NormalTok{(f\_data) }\OtherTok{\textless{}{-}}\NormalTok{ f\_columns}
   \FunctionTok{assign}\NormalTok{(titles, f\_data)}
\NormalTok{   table\_f }\OtherTok{\textless{}{-}} \FunctionTok{append}\NormalTok{(table\_f, }\FunctionTok{list}\NormalTok{(f\_data))}
   \FunctionTok{names}\NormalTok{(table\_f[i]) }\OtherTok{\textless{}{-}}\NormalTok{ titles}
\NormalTok{  \}}
\NormalTok{\}}
\end{Highlighting}
\end{Shaded}

\begin{verbatim}
## Error in f_data %>% separate(Date, c("Date", "Time"), " "): could not find function "%>%"
\end{verbatim}

\begin{Shaded}
\begin{Highlighting}[]
\DocumentationTok{\#\#removes all the null entries in both lists and then names each data frame by its file name}
\NormalTok{table\_f }\OtherTok{\textless{}{-}}\NormalTok{ table\_f[}\SpecialCharTok{{-}}\FunctionTok{which}\NormalTok{(}\FunctionTok{lapply}\NormalTok{(table\_f,is.null) }\SpecialCharTok{==}\NormalTok{ T)]}
\NormalTok{f\_titles }\OtherTok{\textless{}{-}}\NormalTok{ f\_titles[}\SpecialCharTok{{-}}\FunctionTok{which}\NormalTok{(}\FunctionTok{lapply}\NormalTok{(f\_titles,is.null) }\SpecialCharTok{==}\NormalTok{ T)]}
\FunctionTok{names}\NormalTok{(table\_f) }\OtherTok{\textless{}{-}}\NormalTok{ f\_titles}
\end{Highlighting}
\end{Shaded}

\begin{verbatim}
## Error in names(table_f) <- f_titles: 'names' attribute [1] must be the same length as the vector [0]
\end{verbatim}

\begin{Shaded}
\begin{Highlighting}[]
\DocumentationTok{\#\#Joining the particle data and flow data by the data and time}
\ControlFlowTok{for}\NormalTok{ (i }\ControlFlowTok{in} \DecValTok{1}\SpecialCharTok{:}\FunctionTok{length}\NormalTok{(table\_p)) \{}
\NormalTok{   table\_p[[i]] }\OtherTok{\textless{}{-}} \FunctionTok{left\_join}\NormalTok{(table\_p[[i]], table\_f[[i]], }\AttributeTok{by =} \FunctionTok{c}\NormalTok{(}\StringTok{"Date"}\NormalTok{))}
\NormalTok{   Date }\OtherTok{\textless{}{-}}\NormalTok{ table\_p[[i]]}\SpecialCharTok{$}\NormalTok{Date}
\NormalTok{   table\_p[[i]] }\OtherTok{\textless{}{-}}\NormalTok{ (table\_p[[i]][ , }\DecValTok{2}\SpecialCharTok{:}\DecValTok{10}\NormalTok{])}\SpecialCharTok{*}\NormalTok{(table\_p[[i]][ , }\DecValTok{11}\NormalTok{])}
\NormalTok{   table\_p[[i]] }\OtherTok{\textless{}{-}} \FunctionTok{cbind}\NormalTok{(Date, table\_p[[i]])}
   
\NormalTok{\}}
\end{Highlighting}
\end{Shaded}

\begin{verbatim}
## Error in left_join(table_p[[i]], table_f[[i]], by = c("Date")): could not find function "left_join"
\end{verbatim}

I left off in the splitting to bins chunk below. I have been trying
different ways of going about doing it but I cannot seem to get what we
are looking for.

\begin{Shaded}
\begin{Highlighting}[]
\NormalTok{bins }\OtherTok{\textless{}{-}} \FunctionTok{list}\NormalTok{()}


\NormalTok{data }\OtherTok{\textless{}{-}} \FunctionTok{lapply}\NormalTok{(table\_p, }\ControlFlowTok{function}\NormalTok{(x)\{}
\NormalTok{  bins[}\DecValTok{1}\NormalTok{] }\OtherTok{\textless{}{-}} \FunctionTok{c}\NormalTok{(}\FunctionTok{list}\NormalTok{(x[, }\DecValTok{2}\NormalTok{]), bins)}
\NormalTok{  bins[}\DecValTok{2}\NormalTok{] }\OtherTok{\textless{}{-}} \FunctionTok{c}\NormalTok{(}\FunctionTok{list}\NormalTok{(x[, }\DecValTok{3}\NormalTok{]), bins)}
\NormalTok{  bins[}\DecValTok{3}\NormalTok{] }\OtherTok{\textless{}{-}} \FunctionTok{c}\NormalTok{(}\FunctionTok{list}\NormalTok{(x[, }\DecValTok{4}\NormalTok{]), bins)}
\NormalTok{  bins[}\DecValTok{4}\NormalTok{] }\OtherTok{\textless{}{-}} \FunctionTok{c}\NormalTok{(}\FunctionTok{list}\NormalTok{(x[, }\DecValTok{5}\NormalTok{]), bins)}
\NormalTok{  bins[}\DecValTok{5}\NormalTok{] }\OtherTok{\textless{}{-}} \FunctionTok{c}\NormalTok{(}\FunctionTok{list}\NormalTok{(x[, }\DecValTok{6}\NormalTok{]), bins)}
\NormalTok{  bins[}\DecValTok{6}\NormalTok{] }\OtherTok{\textless{}{-}} \FunctionTok{c}\NormalTok{(}\FunctionTok{list}\NormalTok{(x[, }\DecValTok{7}\NormalTok{]), bins)}
\NormalTok{  bins[}\DecValTok{7}\NormalTok{] }\OtherTok{\textless{}{-}} \FunctionTok{c}\NormalTok{(}\FunctionTok{list}\NormalTok{(x[, }\DecValTok{8}\NormalTok{]), bins)}
\NormalTok{  bins[}\DecValTok{8}\NormalTok{] }\OtherTok{\textless{}{-}} \FunctionTok{c}\NormalTok{(}\FunctionTok{list}\NormalTok{(x[, }\DecValTok{9}\NormalTok{]), bins)}
  
\NormalTok{\}}
\NormalTok{)}
\end{Highlighting}
\end{Shaded}

everything below here is to later be used inside the loop above

\#\#////////////////////////////////////Dont run below code until the
data frames are joined correctly
\#\#////////////////////////////////////Will need to add a loop to go
through all runs for each filter to plot

\#\#plotting the data as a scatterplot ggplot(final\_data, aes(Date,
value, col=variable)) + geom\_point() + stat\_smooth() +
ggtitle(``Particle counts from 2/19/21-2/20/21'') + \# for the main
title xlab(``Date'') + \# for the x axis label ylab(``Counts'') + \# for
the y axis label scale\_x\_discrete(breaks = c(2/19/21, 2/20/21,\\
2/20/21 ))

\end{document}
