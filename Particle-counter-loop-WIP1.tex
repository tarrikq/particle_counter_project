% Options for packages loaded elsewhere
\PassOptionsToPackage{unicode}{hyperref}
\PassOptionsToPackage{hyphens}{url}
%
\documentclass[
]{article}
\usepackage{amsmath,amssymb}
\usepackage{lmodern}
\usepackage{ifxetex,ifluatex}
\ifnum 0\ifxetex 1\fi\ifluatex 1\fi=0 % if pdftex
  \usepackage[T1]{fontenc}
  \usepackage[utf8]{inputenc}
  \usepackage{textcomp} % provide euro and other symbols
\else % if luatex or xetex
  \usepackage{unicode-math}
  \defaultfontfeatures{Scale=MatchLowercase}
  \defaultfontfeatures[\rmfamily]{Ligatures=TeX,Scale=1}
\fi
% Use upquote if available, for straight quotes in verbatim environments
\IfFileExists{upquote.sty}{\usepackage{upquote}}{}
\IfFileExists{microtype.sty}{% use microtype if available
  \usepackage[]{microtype}
  \UseMicrotypeSet[protrusion]{basicmath} % disable protrusion for tt fonts
}{}
\makeatletter
\@ifundefined{KOMAClassName}{% if non-KOMA class
  \IfFileExists{parskip.sty}{%
    \usepackage{parskip}
  }{% else
    \setlength{\parindent}{0pt}
    \setlength{\parskip}{6pt plus 2pt minus 1pt}}
}{% if KOMA class
  \KOMAoptions{parskip=half}}
\makeatother
\usepackage{xcolor}
\IfFileExists{xurl.sty}{\usepackage{xurl}}{} % add URL line breaks if available
\IfFileExists{bookmark.sty}{\usepackage{bookmark}}{\usepackage{hyperref}}
\hypersetup{
  pdftitle={Untitled},
  pdfauthor={Tarrik Quneibi},
  hidelinks,
  pdfcreator={LaTeX via pandoc}}
\urlstyle{same} % disable monospaced font for URLs
\usepackage[margin=1in]{geometry}
\usepackage{graphicx}
\makeatletter
\def\maxwidth{\ifdim\Gin@nat@width>\linewidth\linewidth\else\Gin@nat@width\fi}
\def\maxheight{\ifdim\Gin@nat@height>\textheight\textheight\else\Gin@nat@height\fi}
\makeatother
% Scale images if necessary, so that they will not overflow the page
% margins by default, and it is still possible to overwrite the defaults
% using explicit options in \includegraphics[width, height, ...]{}
\setkeys{Gin}{width=\maxwidth,height=\maxheight,keepaspectratio}
% Set default figure placement to htbp
\makeatletter
\def\fps@figure{htbp}
\makeatother
\setlength{\emergencystretch}{3em} % prevent overfull lines
\providecommand{\tightlist}{%
  \setlength{\itemsep}{0pt}\setlength{\parskip}{0pt}}
\setcounter{secnumdepth}{-\maxdimen} % remove section numbering
\ifluatex
  \usepackage{selnolig}  % disable illegal ligatures
\fi

\title{Untitled}
\author{Tarrik Quneibi}
\date{5/20/2021}

\begin{document}
\maketitle

\#\#install packages \#\#install.packages(ggplot2)
\#\#install.packages(reshape2) \#\#install.packages(dplyr)
\#\#install.packages(tidyr) \#\#install.packages(chron)
\#\#install.packages(anytime) \#\#install.packages(lubridate)
\#\#install.packages(tidyverse)

\#load necessary libraries library(ggplot2) library(reshape2)
library(dplyr) library(tidyr) library(chron) library(anytime)
library(lubridate) library(tidyverse)

\#\#setting working directory for the particle counter, calling each csv
file into a list setwd(``U:/public/ADMIN/WQIntern TQuneibi/R
code/particle counter data/'')

\#\#reading in all of the csv files in the directory file\_list
\textless- list.files(pattern = ``.csv'') df\_list \textless- list()
file\_titles \textless- list() bins \textless- c(``Date'', ``Time''
,``Bin 1'', ``Bin 2'', ``Bin 3'', ``Bin 4'', ``Bin 5'', ``Bin 6'', ``Bin
7'', ``Bin 8'', ``Bin 9'') i=0 \#\#this loop takes each file from the
file list and sorts it between patricle and flow data. for (file in
file\_list)\{ i=i+1 if(!grepl(``Flow'', file))\{ \#\#removes the .csv
from the end of each file name titles \textless- substring(file, 1,
nchar(file) - 4) file\_titles{[}i{]} \textless- titles \#\#reading in
the file p\_data \textless- read.csv(file) p\_data \textless-
p\_data{[}, 2:12{]} colnames(p\_data) \textless- bins \#\#combines the
date and time columns back into a single column with the correct format.
\#\#Also changes the class from CHAR to Date-Time
p\_data\(Date <- as.POSIXct(paste(p_data\)Date,
p\_data\(Time), format="%m/%d/%Y %H:%M") ##deleting the old time column  p_data[ , c('Time')] <- list(NULL) ##the time was rounded to the nearest 15 minute mark so that it could be accurately joined with the flow data  p_data
\)Date \textless-
round\_date(p\_data\(Date, unit = "15 minutes") ##combines the file name with the data associated with it to create a variable  assign(titles, p_data)  df_list <- append(df_list, list(p_data))  names(df_list[[i]]) <- file_titles[i]  }  else { ##removes the .csv from the end of each file name  titles <- substring(file, 1, nchar(file) - 4)  file_titles[i] <- titles ##reading each file  f_data <- read.csv(file)  f_data <- f_data[ ,1:3] ##takes every third row so that only every 15 minutes is considered  f_data = f_data[seq(1, nrow(f_data), 3), ] ##changes the first column name to date  colnames(f_data)[1] <- "Date" ##seperates the date and time into two columns  f_data <- f_data %>% separate(Date, c("Date", "Time"), " ") #changes the date to a new format which matches the particle counter date format  f_data
\)Date \textless-
as.Date(f\_data\(Date, format = "%m/%d/%y") ##combines the date and time columns back into a single column with the correct format. ##Also changes the class from CHAR to Date-Time  f_data
\)Date \textless- as.POSIXct(paste(f\_data\(Date, f_data\)Time),
format=``\%Y-\%m-\%d \%H:\%M'') \#\#deleting the old time column
f\_data{[} , c(`Time'){]} \textless- list(NULL) \#\#the time was rounded
to the nearest 15 minute mark so that it could be accurately joined with
the particle data f\_data\(Date <- round_date(f_data\)Date, unit = ``15
minutes'')

f\_data{[} , 2{]} \textless- ifelse(f\_data{[} ,2{]} \textgreater{}
0.01, 1, 0) \#\#combines the file name with the data associated with it
to create a variable assign(titles, f\_data) df\_list \textless-
append(df\_list, list(f\_data)) names(df\_list{[}{[}i{]}{]}) \textless-
file\_titles{[}i{]} \} \}

\begin{verbatim}
\end{verbatim}

everything below here is to later be used inside to loop above ```

\#\#Joining the particle data and flow data by the data and time
final\_data.join \textless- left\_join(p\_data, f\_data, by =
c(``Date''))

\#\#////////////////////////////////////Dont run below code until the
data frames are joined correctly
\#\#////////////////////////////////////Will need to add a loop to go
through all runs for each filter to plot

\#Multiply flow data and particle data to get relevant data counts
\textless- final\_data.join{[} ,5:13{]} flow \textless-
final\_data.join{[} , 30{]} relevent\_data \textless- counts*flow

\#\#add the date back in to the data frame relevent\_data.time
\textless- cbind(p\_data\$Date, relevent\_data)
colnames(relevent\_data.time) \textless-
c(``Date'',``Bin1'',``Bin2'',``Bin3'',``Bin4'',``Bin5'',``Bin6'',``Bin7'',``Bin8'',``Bin9'')
final\_data \textless- melt(relevent\_data.time,id.vars=``Date'' )

\#\#plotting the data as a scatterplot ggplot(final\_data, aes(Date,
value, col=variable)) + geom\_point() + stat\_smooth() +
ggtitle(``Particle counts from 2/19/21-2/20/21'') + \# for the main
title xlab(``Date'') + \# for the x axis label ylab(``Counts'') + \# for
the y axis label scale\_x\_discrete(breaks = c(2/19/21, 2/20/21,\\
2/20/21 ))

\end{document}
